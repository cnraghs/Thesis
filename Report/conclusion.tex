\chapter{Conclusion}
As part of a closed loop DC current transformer, it was required to design a programmable Operational Transconductance Amplifier that generates a current of $\pm$15mA and drives a resistive load. Different topologies of OTAs were considered and evaluated by weighing their pros and cons with respect to the specifications and the conventional current mirror based OTA was chosen. A high output current transconductance amplifier was designed and implemented using the XT018 SOI technology from XFAB. The amplifier consists of two stages - 
\begin{enumerate}
\item A programmable operational transconductance amplifier that is based on conventional current mirror technique. The output voltage is controlled by an external voltage that indrectly controls the bias current.
\item An operational amplifier based on the two-stage Miller compensation technique is used as a voltage buffer. This stage produces high currents due the bulky nature of the common source amplifier.
\end{enumerate}
Using the $G_m/I_D$ methodology, the dimensions of the differential pair were set at an optimal value in moderate saturation for both the stages of the system. To keep the variation of the DC level at the output of the first stage to a minimum, the bias currents were chosen in the milliAmpere range and the output shell of the OTA was designed to be assymetric. The OP AMP was designed to have a wide ICMR because the output swing of the OTA was high. The OP AMP was also designed to produce a high current that drives a resistive load. The programmable parameter - the external bias voltage is swept from 150mV to 700mV and the output is measured across a 50$\Omega$ load resistor. The range of currents produced by this system is from $\pm$15mA to $\pm$30mA. To further control the variation in the DC level at the output of the first stage, the design is modified in such a way that the variation is kept to a minimum. Another design approach is presented with a fixed external voltage, a fixed bias current but a variable load resistor. In this case, the same range of output current is obtained for an external bias voltage of 450mV and the load varying from 35$\Omega$ to 70$\Omega$. The advantage of this approach over the previous approach is the fact that the DC Bias voltage at the output does not vary with variation in the programmable resistor.
\vfill
\clearpage

\section{Summary of Results}

\begin{table} [H]
\centering
\begin{tabular}{@{}cccc@{}}
\toprule
Parameter						& Value/Specification			& Programmable Voltage			& Programmable Resistor	\\ \midrule
Circuit Design					& Programmability				& External Votlage $V_{bias}$	& Load Resistor $R_L$	\\
Transconductance Gain(Gm)		& 75 .. 140 mA/V				& 158.8 .. 298.5 mA/V			& 170 .. 338.9 mA/V		\\
Linear Input Voltage Range		& $\pm$200 mV					& $\pm$100 mV					& $\pm$100 mV			\\
Output Current Range			& $\pm$15 mA 					& +18 mA .. -13 mA				& +15 mA .. -18 mA		\\
Bandwidth						& 10 MHz						& 13.8 MHz						& 13.13 MHz				\\
Slew Rate						& $\pm$900 V/$\mu$s				& $\pm$10.82 V/$\mu$s			& $\pm$10.28 V/$\mu$s	\\
Rise/Fall time					& 4.4 ns						& 22.7 ns						& 23.5 ns				\\
Input Referred Noise			& 3 nV/$\sqrt{Hz}$ @few KHz		& 46.64 nV/$\sqrt{Hz}$ @1MHz	& 36.79 nV/$\sqrt{Hz}$ @1MHz\\
Input Impedance					& 0.5 M$\Omega$					& 3.394 M$\Omega$				& 3.436 M$\Omega$		\\
Output Impedance				& 55 K$\Omega$					& 0.9415 $\Omega$				& 0.953 $\Omega$		\\
HD2								& Less than -75 dBc				& -34.72 dBc					& -34.6 dBc				\\
HD3								& Less than -80 dBc				& -33.61 dBc					& -39.4 dBc				\\
Open Loop Voltage Gain			& Not less than+5 V/V			& 8.41 V/V						& 12.9 V/V				\\
PSRR							& $\pm$20 $\mu$A/V				& 97.76 $\mu$A/V				& 67.32 $\mu$A/V		\\
\bottomrule
\end{tabular}
\caption{Specification vs Results}
\label{tab:Results}
\end{table}

Table.\ref{tab:Results} summarizes the results of the overall system and compares it with the actual specificaitons. Due to varying bias currents of the OTA, the gain keeps varying too. And to avoid saturation, the gain must be low. OTAs in general have very high gain than a conventional OP AMP. To avoid saturation at the output, the input voltage range is reduced to 100mV instead of 200mV. And consequently, the transconductance parameter is double of what was required. Another consideration to lower the linear input voltage range is the input referred noise. A lower value of linear input voltage range allows to have a higher gain by increasing the dimensions of the differential pair. This decreases the input referred noise. The other specifications are picked from the commercial OTA OPA860 (citation), which is a BJT based OTA with a low output impedance buffer. The power supply for the commercial OTA was observed to be +5V and -5V. In this work, the power supplies to be used are -2.5V and 2.5V. So the headroom for the transistors to utilize is much lesser. The OTA being used in an open loop implies that the gain cannot be high.  So there is a trade off  between input referred noise and the gain. Consequently the operating bandwidth gets efected. The key thing here is to note the output impedance requirement. Ideally, current sources have infinite output impedance. Practially however, for an namplifier with an output buffer, a high impedance at the output nodes is impractical to achieve.

\section{Outlook}

The amplifier designed has been simulated at all corners. The OTA can be used to test the Closed loop NDCCT. The system needs optimization to better meet the specifications. But this needs more research, and implementation of sub-circuits like slew rate enhancer, current feedback amplifier, and possibly an emitter follower to amplify the current further if necessary.